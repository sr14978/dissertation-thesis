\documentclass[12pt,a4paper]{article}
\usepackage[latin1]{inputenc}
\usepackage{amsmath}
\usepackage{amsfonts}
\usepackage{amssymb}
\usepackage{hyperref}
\usepackage[left=2cm,right=2cm,top=2cm,bottom=2cm]{geometry}
\author{Samuel Russell}
\title{Innocuous Ciphertexts}
\begin{document}
\maketitle

\section{Contributions}

\begin{itemize}
\item Critically discuss the downfalls of using an FTE system to encode ciphertexts.
\item Build a statistical tool for distinguishing emulated from normal traffic.
\item Build a new scheme that is able to overcome this detection.
\end{itemize}

\pagebreak
\section{Cryptography Background}
After language was invented to share knowledge, humans soon needed a way hide it from those whom it was not intended. The first schemes used word or letter substitutions or the shuffling of characters to make the meaning of a message hard to obtain. These techniques are still used as the basic building blocks today; for example in the AES block cipher, a very popular encryption scheme, of which there are hardware implementations baked in to many of the major computer chips. Early schemes however, relied on the secrecy of the method to protect the privacy of the message which meant that they cannot be reused by others. This lead to the introduction of parametrised constructions that use a key to alter the encryption such as the Vigen\`ere cipher. Modern schemes obey Keroff's principle; that the security of the system only depends on the security of the key, allowing the same encryption method to be used by all.
\begin{itemize}
 \item security through obscurity
 \item perfect security - information-theoretic
 \item conditional security
 \item ciphertexts - formatting
 \item block ciphers
\end{itemize}

\begin{itemize}
 \item hiding existence
 \item stenography
 \item honey encryption
 \item deniable encryption
\end{itemize}

\pagebreak
\section{Format Transforming Encryption}

In 1981, NIST\cite{FIPS74} outlined a mode of DES that was able to restrict the ciphertext of an encryption to a string of a fixed, finite alphabet such as decimal digits $ \Sigma = \{0,1...9\} $. This allows plain texts such as phone numbers of form $ X \in \Sigma^N $ to be encrypted to a ciphertext $ Y = \Sigma^N $ that is also a phone number. Although that FIPS document is now withdrawn it sparked further work such as Brightwell and Smith\cite{DPE}. They first specify the particular use case of encrypting data in place in a database where the ciphertexts must abide by the same type constraints of the database schema. The paper then goes on to describe their  Datatype-Preserving Encryption scheme that again works with DES in a streaming mode and works over an indexed finite alphabet, however the algorithm has redundant elements such as shuffling the alphabet's ordering that are not backed up with reasoning and there is no proof of security. More recently, Black and Rogaway\cite{CAFD} provide three constructions in their paper. They analyse the security of each one against the standard adversary $Adv^{prp}$ and also discuss the practical considerations of scaling.
\\\\
The field has become a highly practical one, with companies such as Voltage moving to take advantage of the concept. They are using it to upgrade older systems by adding a layer of security that since does not change the structure of the data is invisible to the applications that sit on top. Therefore the costs of this retrofitting is vastly reduced since the need to completely redesign the system is usually eliminated. As of 2018, HPE\cite{hp} are still pushing it as part of their SecureData product to help maintain legacy software.
\\\\
To keep up with the work in industry, Bellare et al. wrote a comprehensive paper\cite{fpe} that formally defines the goals of FTE. They did this in the form of adversary based games. They redefine the standard Pseudo Random Permutation (PRP) Security game which challenges an adversary to distinguish the encryption scheme from a random permutation. The paper also gives weaker notions they think are more appropriate. Single Point Indistinguishably (SPI) requires the adversary to be unable to distinguish between the encryption of any chosen plaintext and a randomly selected ciphertext. The Message Recovery (MR) game gives the adversary the distribution of plaintexts and asks for a decryption of a randomly selected ciphertext, given an encryption and verification oracle. Finally they define Message Privacy (MP) as the inability of an adversary to compute a function of a plaintext given the ciphertext; Note in this game the adversary can select the function it hopes it will have most success with. They state the following strength hierarchy for these games, showing PRP does give us the Message Recovery property we ultimately want.
$$ PRP \implies SPI \implies MP \implies MR $$
The first implication is trivial since a random function will produce a random element when given the message. The last last implications is also trivial since MR is just a special cast of MP - where the function is fixed to be the identity. For the proof of $ SPI \implies MP $ the paper shows how an adversary $B$ for SPI can be  constructed using an adversary $A$ for MP with sufficiently large advantage since $B$ is able to run $A$ and satisfy it's calls to the encryption oracle using its own oracle.
\subsection{Rank Then Encipher}
In the second half of the paper\cite{fpe} Bellare et al. then propose a framework called Rank then Encipher. This framework supports and enhances the methods already discussed since it enables any encryption scheme that supports arbitrary size alphabets to be used to encrypt more complex languages. It simplifies the complexity of language features such as a checksum character by maps words to an simple index so that the encryption scheme used can be generic. It does this by partitioning the language in to slices of the words of each length. That is, for the language $\mathcal{X} = \{\mathcal{X}_N : N \in \mathcal{N} \}$ for lengths $\mathcal{N}$, each slice $\mathcal{X}_N = \{ X : X \in \mathcal{X}, \vert X \vert = N  \}$. Now each slice can be arbitrarily ordered as $\mathcal{X} = \{ X_0, X_1, \cdots , X_{n-1} \} $ and a bijective mapping can be defined $rank_N :: \mathcal{X}_N \rightarrow \mathbb{Z}_n$.
Then for the whole language, we can define
$$ rank :: \mathcal{N} \times \mathcal{X} \rightarrow \mathbb{N} \cup \bot $$
$$ rank(N,X) =
\left\{
	\begin{array}{ll}
		rank_N(X)  & \mbox{if } \vert X \vert = N\\
		\bot & \mbox{otherwise} 
	\end{array}
\right.
$$

$$ unrank :: \mathcal{N} \times \mathbb{N} \rightarrow \mathcal{X} \cup \bot $$
$$ unrank(N,i) =
\left\{
	\begin{array}{ll}
		rank_N^{-1}(i)  & \mbox{if } i \in \mathbb{Z}_{\mathcal{X}_N}\\
		\bot & \mbox{otherwise} 
	\end{array}
\right.
$$\\\\
Then if we have an encryption scheme $Enc :: \mathcal{K} \times \mathcal{N} \times \mathbb{N} \rightarrow \mathbb{N}$ that works over arbitrary domains to encrypt values into $\mathbb{Z}_{|\mathcal{X}_N|}$, we can define out scheme $\mathcal{E}_k^N = unrank_N \cdot Enc_k^{|X_N|} \cdot rank_N$. Using this construction, they state how the security properties of the original encryption scheme over the finite domain is inherited by the the rank then encrypt scheme over the language.
\\\\
They provide a ranking and un-ranking algorithm for regular languages, but to make this efficient, pre-computation is needed. For each word length a table has to be created which can be done through dynamic programming.
\\\\
Furthermore, using M\"akinen's\cite{rankcf}, algorithms if a language can be described an unambiguous context free grammar then ranking can be done in linear time and unranking can be done in $O(n \log n)$ time, if we allow $O(n^2)$ time and space preprocessing. This is since the left Szilard language of the grammar can be more easily ranked and unranked. Conversion back to the original grammar is as simple as linearly applying the rules and conversion from a word to it's left Szilard counterpart depends on the efficiency of parsing, which, is not linear, but for unambiguous grammars is still efficient.

\begin{itemize}
 \item rank then encipher approach
 \item security notations
 \item application for proxy transport
 \item downfall
 \item opaque vs innocuous
\end{itemize}

\pagebreak
\section{Motivation}

Tim Berners-Lee says\cite{gard} that since its conception, the world wide web has been built with the principles of openness and privacy. These key values, that also lie at the heart of democracy, can ensure that if a website is made available on web, anyone can connect to it - no matter who they are or where they live. 
\\\\
However, recently, changes have been made which challenge this status quo. These changes are the censorship of certain content on the internet for users in some places like the United Kingdom or in China. The motivations behind this censorship can be well-meaning, for example, the blocking of illegal content such as child pornography with the intention of protecting the innocent lives of children. Further attempts to reduce damage to people's lives was the implementation of search engine de-listing of personal content partitioned by the right-to-be-forgotten movement\cite{rtbf}. This decision sparked much controversy since people believed that this deletion of history could lower the quality of the internet. 
\\\\
However, other cases are on a much larger scale, such as nation states blocking access to content that they deem to be opposing to their political standpoints. This can involve blanket restrictions to foreign websites or restricting sites, which may be hosted inside the nation, but are deemed to perpetuate negative views. Many countries ban hate-speech, or exclude it from their freedom of speech laws\cite{hate} and some countries have Libel rules\cite{libel} which ban spreading of false information. However, there are a number of nations that actually hide information about certain world events or ideologies from their citizens, depriving their civil rights\cite{chincensor}. This is dangerous for the world as it creates environments that can breed hate directed at whomever the censors want.
\\\\
There is no doubt that the western economy and society has benefited greatly with the success of the web with its ability to connect people and share information. What ensures the web's success is it's status as a permissionless space that leads to creativity, innovation and freedom of expression. This allows any new, small, start-up companies and individuals to use this commodity to build services and communities without the need for legal approval or business deals, leaving their projects to be free to grow and prosper. Proposed Net Neutrality regulations would have allowed internet service providers to prioritise certain site's traffic. This does not permit full censorship but even the slowing down of traffic could reduce the usability of certain sites so that users would avoid them. This would fundamentally coerce browser's behaviour to direct them away from certain topics.
\\\\
Taking an existing example; in China, which is a state with among the most stringent internet censorship practices, the American Chamber of Commerce\cite{amcham} says that 4 out of 5 of its member companies report a negative impact on their business from Internet censorship. This gives financial incentives to ensure the internet remains open.

\section{Network Surveillance and Censorship}

I will categorise censorship in two ways. The first kind is entities on the internet such as websites taking down content. This is most poignant when the content has been made by others, such as on a micro-blogging site like Facebook. Although, this is an issue and currently there is debate about the balance of such censorship on these social media platforms, this is not what I will be looking at. This is because the data is gone rather that being blocked. Of course, there are tools that archive content that maybe be later re-tracked or deleted such as The Internet Archive's project: Wayback Machine\cite{wb}, and access to these resources is what the second type of censorship is about.
\\\\
The internet consists of each party's computer equipment connected together with a network of wires (and other mediums) and some management machinery to route communications to their destination.  Communications are packet based and are directed along the connections according to the Internet Protocol (IP)\cite{ip4}. This protocol defines the destination of the packet for the purposes of working out the most appropriate route to follow. It also contains the address of the source so error messages can be directed appropriately. Among other information like version numbers and flags there is also a field that tells the intermediary machinery when to delete the message, but this is only for cases where the destination in unreachable. 
\\\\
The second type of network surveillance and censorship sits at these intermediate points in the network and deals with how these packets are handled. By default IP and the transport layer on top of it (TCP) does not encrypt or sign the payload data. Therefore, it is possible for the intermediary parties to read and/or tamper with the packets and their payloads before they are sent or even dropped from the network. 
\\\\
From their observations of the chinese internet filtering systems, Zittrain and Edelman\cite{edelman2005empirical} deduced several methods that were being used. Filtering on the IP address of a resource can be done by analysing the destination field of the packet. This is good for blocking direct access to static sites since it  is one of the simplest and therefore fastest to implement, though it relies on comprehensive black listing coordination.
\\\\
The most common method of blocking is for the routers to drop these packets, but as Clayton et al. found in their practical experiments\cite{clayton2006ignoring} for protocols that communicate in a streaming fashion over time the network machinery sends TCP reset packets to both parties to interrupt the session and stop communication. This type of attack can be easily countered by ignoring all reset packets since due to how the system is architectured the original packets are allowed through. They also observed a less frequent technique which was the injection of a forged packet with random synchronisation numbers. When this reaches the destination the server detects the incorrect value and validly triggers the reset packet. Now, if the reset packets are ignored then the two real endpoints become out of sync and the communication breaks down. Luckily as it stands, the forged packets are poor imitations and so are easily detectable.
\\\\
Other method used in practice is DNS poisoning, which breaks involves incorrectly resolving URLs for users to redirect them to a Censorship notice or just a random useless address. This only affects the use of URLs and so raw IP address access is completely unaffected, although many sites use them internally so it creates great difficulty and work for users.
\\\\
Finally, packet inspection has become very common place. This can be content based detection such as searching for trigger words in HTTP request paths, or in the html content returned. A good example of this is search terms in search engine requests. Although this only applies to clear unencrypted http traffic and using an encrypted  TLS transport layer hides all this information sites that use this risk having the whole site IP blocked as above. This happened to Wikipedia in China when the site switched to enforcing HTTPS on all traffic.
\\\\
The packet inspection is also done on a protocol basis. Censors are usually tolerant to users accessing webpages over HTTP or video calling over VOIP protocols, however, protocols that allow the transfer of data in a opaque way such as VPN tunnelling, The Onion Router and in general TLS are often blocked completely.

\section{The Onion Router}

One of the most popular tools for evading network surveillance is The Onion Router which wraps packets in multiple layers of encryption and passed them through a relay network which unwraps a layer at a time to anonymize the sender-receiver combination. Since this prevents outsiders from knowing what sites an individual visits, censors such as the Chinese Government have resorted to a complete ban on the service. Tor packets are very recognisable due to their predictable size and structure but some improvement have been made to adjust it's fingerprint to match TLS as much as possible. 

Tor works by allowing users to connect to one of the publicly known entry guard relays however, since these IP addresses are public they are easily detected and interfered with by in-line systems. One way this is done is by sending reset packets to both parties telling them to close the TCP stream.
To circumvent this, users instead connect to private bridges in uncensored zones (such as the US) so that the indented destination of the packets was hidden. However, with protocol fingerprinting by deep packet inspection (DPI) and the use of active probes censors can still catch these unlisted bridges in under 15 minutes of use!
The use of FTE bridges has been able to fool these DPI systems for now since they fool the common protocol classification systems which use regular expressions also. However, as deep learning becomes more prevalent it is known that this will not be enough.

\pagebreak

\bibliography{thesis}{}
\bibliographystyle{plain}

\end{document}